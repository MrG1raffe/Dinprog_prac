\documentclass[16pt]{article}
\usepackage[utf8]{inputenc}
\usepackage[russian]{babel}
\usepackage{a4wide}
\usepackage{graphicx}
\usepackage{amsmath}
\usepackage{amssymb}
\usepackage{mathrsfs}  
\usepackage{amsthm}
\usepackage{import}
\usepackage{xifthen}
\usepackage{pdfpages}
\usepackage{transparent}
\usepackage{multicol}
\usepackage{indentfirst}
\usepackage{bbm}

\newcommand{\incfig}[2]{%
    \def\svgwidth{#2 mm}
    \import{./figures/}{#1.pdf_tex}
}

\newtheorem{Th}{Теорема}
\newtheorem{Lem}{Лемма}
\newtheorem{Def}{Определение}
\newtheorem*{Cons}{Следствие}
\newtheorem{Rem}{Замечание}
\newtheorem{St}{Утверждение}
\newenvironment{Proof}{\par\noindent{\bf Доказательство. \\}}{\hfill$\scriptstyle\blacksquare$}

\DeclareMathOperator{\Arccos}{arccos}
\DeclareMathOperator{\Arctg}{arctg}
\DeclareMathOperator{\Cl}{Cl}
\DeclareMathOperator{\Sign}{sgn}
\DeclareMathOperator*{\Argmax}{Argmax}
\DeclareMathOperator*{\Var}{Var}
\DeclareMathOperator*{\Min}{min}


\newcommand\Real{\mathbb{R}} 
\newcommand\A{(\cdot)} 
\newcommand\Sum[2]{\sum\limits_{#1}^{#2}}
\newcommand\Scal[2]{\langle #1,\, #2 \rangle}
\newcommand\Norm[1]{\left\| #1 \right\|}
\newcommand\Int[2]{\int\limits_{#1}^{#2}}
\newcommand{\Me}{\mathbb{E}}
\newcommand{\I}{\mathbbm{1}}
\newcommand{\Prb}{\mathbb{P}}
\newcommand\Reff[1]{(\ref{#1})}
\newcommand{\deq}{\overset{d}{=}}
\newcommand{\Bor}{\mathcal{B}}

\begin{document}
\thispagestyle{empty}

\begin{center}
\ \vspace{-3cm}

\includegraphics[width=0.5\textwidth]{msu.eps}\\
{\scshape Московский государственный университет имени М.~В.~Ломоносова}\\
Факультет вычислительной математики и кибернетики\\
Кафедра системного анализа

\vfill

{\LARGE Отчёт по практикуму}

\vspace{1cm}

{\Huge\bfseries <<Эллипсоидальные оценки для множества достижимости>>}
\end{center}

\vspace{1cm}

\begin{flushright}
  \large
  \textit{Студент 415 группы}\\
  Д.\,М.~Сотников

  \vspace{5mm}

  \textit{Руководитель практикума}\\
  к.ф.-м.н., доцент И.\,В.~Востриков
\end{flushright}

\vfill

\begin{center}
Москва, 2020
\end{center}

\newpage
\tableofcontents
\newpage
\section{Постановка задачи}
Рассматривается линейная управляемая система:

\begin{equation}\label{task}
\begin{cases}
\dot{x}(t)  = A(t)x(t) + B(t)u(t), \ \ t \in [t_0, t_1],\\
x(t_0) \in \mathcal{E}(x_0, X_0),\\
u(t)\in \mathcal{E}(q(t), Q(t)).
\end{cases}
\end{equation}


Здесь $u(t) \in \Real^m, \, x(t) \in \Real^n,\, x_0 \in \Real^n, \,X_0 \in \Real^{n \times n},\, X_0 = X_0^* > 0, \, A(t) \in \Real^{n \times n}, \,B(t) \in \Real^{n \times m},\, q(t) \in \Real^m, \, Q(t) \in \Real^{m \times m},\, Q(t) = Q^*(t) > 0 $ при любых $t$ из $[t_0,\, t_1]$. Функции $A(t),\, B(t),\,$ $q(t),\, Q(t)$ непрерывны на $[t_0, t_1].$


Необходимо построить:
\begin{enumerate}
	\item Внутренние и внешние эллипсоидальные оценки множества достижимости системы (\ref{task}).
	\item Проекции множества разрешимости на двумерную плоскость.
	\item Трехмерную проекцию множества разрешимости.
	
\end{enumerate}

\section{Внутренние и внешние оценки для суммы эллипсоидов}

Обозначим эллипсоид с центром $q \in \Real^n$ и матрицей конфигурации $Q \in \Real^{n \times n},\ Q = Q^* > 0$ 

$$\mathcal{E}(q, Q) = \{x\colon \Scal{(x - q)}{Q^{-1}(x - q)} \leq 1\}.$$

Нас будут интересовать верхние и нижние оценки $\mathcal{E}_-,\, \mathcal{E}_+$ суммы (по Минковскому) конечного числа эллипсоидов $\mathcal{E}_1, \ldots, \mathcal{E}_n$:
$$\mathcal{E}_- \subset \mathcal{E}_1 + \ldots + \mathcal{E}_n \subset \mathcal{E}_+.$$

\subsection{Внешняя оценка}
Пусть $p_1, \ldots, p_n > 0$. Покажем, что 
$$\mathcal{E}_+ = (p_1 + \ldots + p_n)\left(\dfrac{Q_1}{p_1} + \ldots + \dfrac{Q_n}{p_n}\right)$$
является внешней оценкой.

Действительно,
\begin{multline}\notag
\rho(l|\mathcal{E}_+)^2 = \Sum{i=1}{n}\Scal{l}{Q_il} + \sum_{i > j} \left(\dfrac{p_j}{p_i}\Scal{l}{Q_il} + \dfrac{p_i}{p_j}\Scal{l}{Q_jl}\right) \geq \\ \geq\Sum{i=1}{n}\Scal{l}{Q_il} + 2\sum_{i > j}\sqrt{\Scal{l}{Q_il}\Scal{l}{Q_jl}} = \rho(l|\mathcal{E}_1 + \ldots + \mathcal{E}_n)^2.
\end{multline}

Равенство здесь достигается тогда и только тогда, когда $p_i = \sqrt{\Scal{l}{Q_il}},\ i = 1, \ldots, n$. Таким образом, опорный вектор $\mathcal{E}_+$ по направлению $l$ совпадает с опорным ветором $\mathcal{E}_1 + \ldots + \mathcal{E}_n$, и потому
$$ \mathcal{E}_1 + \ldots + \mathcal{E}_n = \bigcap_{l \in S_1} \mathcal{E}_+(l).$$

\subsection{Внутренняя оценка}

Пусть матрица конфигурации для $\mathcal{E}_-$
$$Q_- = Q_*^*Q_*, \quad Q_* = \Sum{i=1}{n}S_iQ_i^\frac12,$$
где $S_i$ --- ортогональные матрицы.

С помощью неравенства Коши--Буняковского можно показать, что $Q_-$ действительно является внутренней оценкой, и, более того $\rho(l|\mathcal{E}_-) = \rho(l|\mathcal{E}_1 + \ldots + \mathcal{E}_n)$ в том и только том случае, когда $S_iQ_i^\frac12l = \lambda_iS_1Q_1^\frac12l$.

\section{Эллипсоидальные оценки для интеграла}
Рассмотрим выражение 
$$I(t) = \mathcal{E}_0(0, Q_0) + \int\limits_{t_0}^{t}\mathcal{E}(0, Q(\tau))d\tau$$
--- интеграл от эллипсоида $\mathcal{E}(0, Q(t)))$ по переменной $\tau \in [t_0, t]$. Введем разбиение $\{\tau_i\}_{i = 1}^{N}$ отрезка интегрирования $[t_0, t]$ и будем представлять интеграл в виде сумм 
$$
I(t) = \lim\limits_{N \rightarrow \infty}I_N,
$$
где $I_N = \mathcal{E}_0 + \sum\limits_{i = 1}^{N}\sigma \mathcal{E}_i,\quad \sigma = \dfrac{t - t_0}{N}, \quad \mathcal{E}_{i} = \mathcal{E}(0, Q(\tau_i)).
$

\subsection{Внешняя оценка}
Матрица конфигурации внешней оценки интегральной суммы находится по формуле
\begin{multline}\notag
Q_+^N = \left( p_0 + \sum\limits_{i = 1}^{N}p_i \right)\left( \dfrac{Q_0}{p_0} + \sum\limits_{i = 1}^{N} \dfrac{Q_i \sigma^2}{p_i}\right) = \left\{ p_0 = \langle l, Q_0 l \rangle ^{\frac{1}{2}},\ p_i = \sigma \langle l, Q_i l \rangle ^{\frac{1}{2}}\right\} =\\
= \left( p_0 + \sum\limits_{i = 1}^{N}\sigma\langle l, Q_il\rangle^{\frac{1}{2}} \right)\left( \dfrac{Q_0}{p_0} + \sum\limits_{i = 1}^{N} \sigma \dfrac{Q_i}{\langle l, Q_il\rangle^{\frac{1}{2}}}\right).
\end{multline}

При стремлении $N$ к бесконечности получим следующую матрицу конфигурации для внешней оценки интеграла:
$$Q(\tau) = \left( p_0 + \int\limits_{t_0}^{t} p(\tau)d\tau\right)\left( \dfrac{Q_0}{p_0} + \int\limits_{t_0}^{t}\dfrac{Q(\tau)}{p(\tau)}d\tau\right).$$

Таким образом, интеграл аппроксимируется пересечением эллипсоидов по различным направлениям $l$, то есть $I(t) = \bigcap\limits_{\Norm{l} = 1} \mathcal{E}_+(l)$.

\subsection{Внутренняя оценка}
Аналогичными рассуждениями приходим к тому, что матрица конфигурации нижней оценки для интеграла имеет вид
$$Q_- = Q_*^*Q_*, \quad Q_* = S_0Q_0^\frac12 + \Int{t_0}{t}S(\tau)Q^\frac12(\tau)d\tau,$$

причем оценка будет точной по направлению $l$, если 
$$S_0Q_0^\frac12l = \lambda(\tau)S(\tau)Q^\frac12(\tau)l.$$

В качестве $S_0$ далее будем использовать единичную матрицу.

\section{Эллипсоидальное оценивание множества достижимости}
Будем дополнительно предполагать, что в задаче (\ref{task}) выполнено $\ B(t)Q(t)B^*(t) > 0$ при любых $t \in [t_0,\, t]$. Оценим множество достижимости $\mathcal{X}[t]$.

Найдем траекторию $x(t)$ по формуле Коши:

$$x(t) = X(t, t_0)x(t_0) + \int\limits_{t_0}^{t} X(t, \tau)B(\tau)u(\tau)d\tau,$$

где $X(t, \tau)$ --- фундаментальная матрица, полученная как решение задачи Коши:

\begin{equation}\label{fund}
\left\{
\begin{aligned}
&\dfrac{\partial X(t, \tau)}{\partial \tau} = -X(t, \tau)A(\tau),\\
&X(t, t) = I.
\end{aligned}
\right.
\end{equation}



Множество достижимости можно представить следующим образом:
$$
\mathcal{X}[t] = X(t, t_0)\mathcal{E}(x_0, X_0) + \int\limits_{t_0}^{t}X(t, \tau)B(\tau)\mathcal{E}(q(\tau), Q(\tau))d\tau.
$$

\begin{St}
Пусть $\mathcal{E}(p, P)$ --- эллипсоид с центром $p$ и матрицей конфигурации $P$, a $B$ --- невырожденная квадратная матрица тогда $B\mathcal{E}(p, P) = \mathcal{E}(Bp, BPB^*)$.
\end{St}
\begin{Proof}
Пусть $x \in \mathcal{E}(p, P)$, что равносильно $\Scal{x}{P^{-1}x}  \leq 1$.

Рассмотрим теперь $B\mathcal{E}(p, P) = \mathcal{E}(q, Q)$, для него будет верно 
$$\langle B^{-1}x, P^{-1}B^{-1}x\rangle = \langle x, \left(B^{-1}\right)^* P^{-1} B^{-1}x \rangle \leqslant 1.$$
Следовательно, $Q^{-1} = \left(B^{-1}\right)^* P^{-1} B^{-1}$ и $Q = B P B^*$, что и требовалось доказать.
\end{Proof}


Из утверждения следует, что множество $\mathcal{X}[t]$ может быть записано в виде:
\begin{multline}\notag
\mathcal{X}[t] = \mathcal{E}(X(t, t_0)x_0,\, X(t, t_0)X_0 X^*(t, t_0)) + \\+\int\limits_{t_0}^{t} \mathcal{E}(X(t, \tau)B(\tau) q(\tau),\, X(t, \tau) B(\tau) Q(\tau) B^*(\tau) X^*(t, \tau))d\tau.
\end{multline}

Таким образом, задача нахождения множества достижимости сводится к задаче интегрирования эллипсоидов, рассмотренной в предыдущем разделе. Построим внешнюю и внутреннюю элипсоидальные оценки $\mathcal{X}[t]$.


Обозначим внешнюю эллипсоидальную оценку $\mathcal{E}(x_{+}(t), X_{+}(t))$. Параметры задаются формулами
$$
x_+(t) = X(t, t_0)x_0 + \Int{t_0}{t}X(t, \tau)B(\tau)q(\tau)d\tau,
$$

$$
X_{+}(t) = \left( p_0 + \int\limits_{t_0}^{t} p(\tau)d\tau\right)\left(\dfrac{X(t, t_0)X_0 X^*(t, t_0)}{p_0} + \int\limits_{t_0}^{t} \dfrac{X(t, \tau)B(\tau)Q(\tau) B^*(\tau) X^*(t, \tau)}{p(\tau)}d\tau\right),
$$ 

где 
$$
\begin{aligned} 
&p_0 = \langle l, X(t, t_0)X_0 X^*(t, t_0)l \rangle^{\frac{1}{2}},\\
&p(\tau) =\langle l, X(t, \tau) B(\tau) Q(\tau) B^*(\tau) X^*(t, \tau) l\rangle^{\frac{1}{2}}.
\end{aligned}
$$

Выбранные таким образом $p$ позволяют получить оценку, точную в направлении $l$.

Внутренняя оценка получается аналогичным образом.
$${X}_-(t) = Q_*^*(t)Q_*(t),$$
$$Q_*(t) = X_0^\frac12X^*(t, t_0) + \Int{t_0}{t}S(\tau)Q^{\frac12}(\tau)B^*(\tau)X^*(t, \tau)d\tau.$$

Оценка будет точной по направлению $l$, если для любых $\tau$ будет выполнено
\begin{equation}\label{eq_cond}
\lambda(\tau)X_0^\frac12X^*(t, t_0)l = S(\tau)Q^{\frac12}(\tau)B^*(\tau)X^*(t, \tau)l.
\end{equation}
Поскольку $S(\tau)$ является ортогональной матрицей, 
$$\lambda(\tau) = \dfrac{\Norm{Q^{\frac12}(\tau)B^*(\tau)X^*(t, \tau)l}}{\Norm{X_0^\frac12X^*(t, t_0)l}}.$$

Найдем теперь $S(\tau)$ из сингулярного разложения векторов
$$\lambda(\tau)X_0^\frac12X^*(t, t_0)l = U_1d_1V_1,$$
$$Q^{\frac12}(\tau)B^*(\tau)X^*(t, \tau)l = U_2d_2V_2.$$
Здесь $U_1, U_2$ --- ортогональные матрицы, $d_1 = (d_{11}, 0, \ldots, 0)^T,\, d_2 = (d_{21}, 0, \ldots, 0)^T$,
$V_1, V_2 \in \{-1,\, 1\}$. Так как нормы векторов совпадают, и $d_1 \geq 0, \, d_2 \geq 0$, верно $d1 = d2$.

Теперь, если выбрать $S(\tau) = \dfrac{V_1}{V_2}U_1U_2^*$, равенство (\ref{eq_cond}) будет выполнено, а матрица $S(\tau)$ окажется ортогональной.

\section{Описание алгоритма}

Верхние и нижние эллипсоидальные оценки вычисляются по формулам для $\mathcal{X}_+(t),\, \mathcal{X}_-(t)$,
которые были описаны в предыдущем разделе. Фундаментальная матрица как функция $X(t, \cdot)$ находится численно из системы (\ref{fund}).

Для нахождения проекции множества достижимости перебираются направления $l$ из шара соответствующей размерности, для каждого из них вычисляется эллипсоидальная оценка, точная по этому направлению. По оценке вычисляется опорный вектор по направлению $l$. Объединение таких опорных векторов представляет собой численную аппроксимацию проекции множества достижимости.

\section{Примеры работы алгоритма}

Построение внутренних и внешних эллипсоидальных оценок приведено на рис. 1, 2. 

На рис. 3 показана проекция множества достижимости трехмерной системы на координатную плоскость $Ox_2x_3$.

На рис. 4 изображено само множество достижимости этой системы в момент $t = 1$.

\newpage
\begin{figure}[h]
	\center
    \includegraphics[scale=0.75]{int.pdf}
    \caption{Внутренние оценки.}
\end{figure}

\begin{figure}[h]
	\center
    \includegraphics[scale=0.75]{ext.pdf}
    \caption{Внешние оценки.}
\end{figure}

\newpage
\begin{figure}[h]
	\center
    \includegraphics[scale=0.8]{tube.pdf}
    \caption{Проекция множества достижимости на плоскость.}
\end{figure}

\begin{figure}[h]
	\center
    \includegraphics[scale=0.8]{set.pdf}
    \caption{Множество достижимости.}
\end{figure}
\end{document}